\documentclass[]{article}
\usepackage[utf8]{inputenc}
\usepackage{polski}
\usepackage{listings}
\usepackage[usenames,dvipsnames]{xcolor}
\usepackage{geometry}
\usepackage{subcaption}
\usepackage{graphicx}
\usepackage{amsmath}
\usepackage{amssymb}
\usepackage{enumerate}
\usepackage[font=small]{caption}
\usepackage[ruled,noend]{algorithm2e}
\SetAlgorithmName{Algorytm}{algorytm}{Algorytmy}
\DeclareGraphicsExtensions{.png}
\graphicspath{ {./} }
\geometry{
	a4paper,
	left=25mm,
	right = 25mm,
	top=25mm,
}
%%\hyphenchar\font=-1

\title{
	Sprawozdanie \\
	\large 
	Obliczenia naukowe - lista 4}
\author{Kamil Król}
\date{244949}


\begin{document}
	
	\maketitle
	
	\section*{Zadanie 1}
	Celem tego zadania było napisanie funkcji w języku Julia, która oblicza ilorazy różnicowe. Dodatkowym wymaganiem było nieużywanie tablicy dwuwymiarowej.\\
	\textbf{Dane:}
	\begin{enumerate}[]
		\item \texttt{x} -- wektor długości $n+1$ zawierający węzły $x_0, \ldots, x_n$,
		\item \texttt{f} -- wektor długości $n+1$ zawierający wartości interpolowanej funkcji w poprzednio podanych\\ węzłach tj. $f(x_0), \ldots, f(x_n)$.
	\end{enumerate}
	\textbf{Oczekiwany wynik:}
	\begin{enumerate}[]
		\item fx –- wektor długości n + 1 zawierający obliczone ilorazy różnicowe
	\end{enumerate}
	\textbf{Opis:}\\
	Najpierw przyjrzyjmy się temu w jaki sposób można obliczyć ilorazy różnicowe.
	Poniżej znajduje się wzór rekurencyjny pozwalający na obliczenie ilorazu różnicowego k-tego rzędu.
	\begin{enumerate}[]
		\item dla $k = 0$  $$f[x_i] = f(x_i),$$
		\item dla $k = 1$  $$f[x_i,x_j] = \frac{f(x_j) - f(x_i)}{x_j - x_i},$$  
		\item dla $k > 1$  $$f[x_i,x_{i+1}, \ldots, x_{i+k}] = \frac{f[x_{i+1},x_{i+2}, \ldots, x_{i+k}] - f[x_{i}, x_{i+1}, \ldots, x_{i+k-1}]}{x_k - x_i}.$$
	\end{enumerate}
	Ważnym faktem jest to, że wartość ilorazu różnicowego nie zależy od kolejności węzłów ($x_i$). Kolejny użyteczny fakt to to, że 
	znajomość węzłów $x_i$ i wartości funkcji $f(x_i)$ (a więc też ilorazów różnicowych zerowego rzędu tj. $f[x_i] = f(x_i)$) pozwala, przy użyciu powyższego wzoru rekurencyjnego, na stworzenie tzw. tablicy ilorazów różnicowych dla wyższych rzędów. Przyjmując, że $d_{ik} = f[x_i,x_{i+1}, \ldots, x_{i+k}]$ można wyrazić ją w następujący sposób:
\begin{table}[!htbp]
	\centering
	\begin{tabular}{cccccc}
		$d_{0,0}$ & $d_{0,1}$ & $d_{0,2}$ & \ldots & $d_{0,k-1}$ & $d_{0,k}$ \\
		$d_{1,0}$ & $d_{1,1}$ & $d_{1,2}$ &\ldots & $d_{1,k-1}$ & \\
		\ldots & \ldots & \ldots & \ldots && \\
		$d_{k-1,0}$ & $d_{k-1,1}$ & $d_{k-1,2}$ &&& \\
		$d_{k-1,0}$ & $d_{k-1,1}$ &&&& \\
		$d_{k,0}$ &&&&& \\
	\end{tabular}
\end{table}

	\noindent Pierwsza intuicja co do zaprogramowania funkcji obliczającej ilorazy różnicowe to użycie macierzy -- tablicy dwuwymiarowej. Zastanówmy się najpierw czy można to zrobić bardziej efektywnie i jakich danych z powyższej tablicy ilorazów różnicowych potrzebujemy. Interesujące dla nas są tylko dane w pierwszym wierszu tej tablicy. Jeśli dodatkowo zauważymy, że każda kolumna zależy tylko i wyłącznie od poprzedniej kolumny możemy zaproponować rozwiązanie używające tablicy jednowymiarowej. W pierwszym kroku powinniśmy zapisać wartości pierwszej kolumny do jednowymiarowej tablicy. Te dane już mamy, ponieważ są to wartości funkcji w danych węzłach. (Przypomnijmy, że $f[x_i] = f(x_i)$). Następnie w każdym kolejnym kroku powinniśmy wpisywać odpowiednie
	wartości z kolejnych kolumn na ostatnie miejsca w tablicy. W rezultacie w naszej tablicy otrzymamy tylko wartości ilorazów z pierwszego wiersza.\\
	\textbf{Pseudokod algorytmu}\\
	\begin{algorithm}[h]
		\DontPrintSemicolon
		\SetKwProg{Fn}{function}{}{}
		\SetKw{KwDownTo}{downto}
		
		\SetKwFunction{ir}{ilorazyRoznicowe}
		\SetKwFunction{len}{length}
		
		\Fn{\ir{$x$,$f$}}{
			\For{$i \gets 1$ \KwTo \len{$f$}} {
				$fx[i] \gets f[i]$\;
			}
			\For{$i \gets 1$ \KwTo \len{$f$}} {
				
				\For{$j \gets$\len{$f$} \KwDownTo $i$} {
					$fx[j] \gets \dfrac{fx[j]-fx[j-1]}{x[j] - x[j-i]}$\;
					
				}
			}
			
			\KwRet $fx$\;
		}
		\caption{Obliczanie ilorazów różnicowych}
	\end{algorithm}
	
	\section*{Zadanie 2}
	
	Celem tego zadania było napisanie funkcji obliczającej wartość wielomianu interpolacyjnego stopnia $n$ w postaci Newtona $N_{n}(x)$ w punkcie $x = t$ za pomocą uogólnionego algorytmu Hornera, która działa w czasie liniowym ($O(n)$). \\
	\textbf{Dane:}
	\begin{enumerate}[]
		\item \texttt{x} -- wektor długości $n+1$ zawierający węzły $x_0, \ldots, x_n$,
		\item \texttt{fx} -- wektor długości $n+1$ zawierający ilorazy różnicowe,
		\item \texttt{t} -- punkt, w którym należy obliczyć wartość wielomianu.
	\end{enumerate}
	\textbf{Oczekiwany wynik:}
	\begin{enumerate}[]
		\item \texttt{nt} -- wartość wielomianu w punkcie $t$
	\end{enumerate}
	\textbf{Opis:}\\
	\noindent Wzór na wielomian interpolacyjny Newtona $N_n$ pokazuje w jaki sposób zależy on od funkcji $f$. Wzór ten można przedstawić używając ilorazów różnicowych:
	$$N_n(x) = \sum_{i=0}^n f[x_0,x_{1}, \ldots, x_{i}] \prod_{j=0}^{i-1}(x-x_j).$$
	Z numerycznego punktu widzenia takie przedstawienie wielomianu interpolacyjnego jest bardzo atrakcyjne. Zauważmy, że w sytuacji kiedy chcielibyśmy dodać nowe węzły $(x_i, y_i)$ możemy to zrobić korzystając ze wcześniej policzonych $d_k = f[x_0,x_1, \ldots, x_k]$. Kluczowa jest tu własność ilorazów różnicowych mówiąca, że wartość ilorazu nie zależy od kolejności węzłów. Kolejną zaletą takiego zapisu jest to, że wartość tego wielomianu możemy łatwo obliczyć korzystając z uogólnionego algorytmu Hornera. Sposób w jaki można to zrobić przedstawiono poniżej.
	\begin{align*}
	&w_n(x) := f[x_0, x_1, \ldots, x_n]& \nonumber \\
	&w_k(x) := w_{k+1}(x-x_k)+ f[x_0, x_1, \ldots, x_k]	\quad(k=n-1, n-2, \ldots, 0)& \nonumber \\
	&N_n(x) = w_0(x) \nonumber \\
	\end{align*}
	\clearpage
	\noindent\textbf{Pseudokod algorytmu}\\
	\begin{algorithm}[h]
		\DontPrintSemicolon
		\SetKw{KwDownTo}{downto}
		\SetKwProg{Fun}{function}{}{}
		\SetKwFunction{LEN}{length}
		\SetKwFunction{F}{warNewton}
		
		\Fun{\F{$x$, $fx$, $t$}} {
			$n \gets \LEN{fx}$\;
			$nt \gets fx[n]$\;
			\For{$i \gets n-1$ \KwDownTo $1$} {
				$nt \gets fx[i] + (t - x[i]) \times nt$\; 		
			}
			\KwRet $nt$\;
		}
		\caption{Obliczanie wartości wielomianu interpolacyjnego w punkcie $t$.}
	\end{algorithm}			
	
	

	\section*{Zadanie 3} 
	
	Celem tego zadania było napisanie funkcji obliczającej współczynniki $a_0,\ldots,a_n$ postaci naturalnej wielomianu interpolacyjnego dla zadanych współczynników $d_0 = f[x_0], d_1 = f[x_0,x_1], \ldots d_n = f[x_0, \ldots, x_n]$ tego wielomianu w postaci Newtona oraz węzłów $x_0, \ldots, x_n$. Ponadto funkcja miała działać w czasie $O(n^2)$.\\
	\textbf{Dane:}
	\begin{enumerate}[]
		\item \texttt{x} -- wektor długości $n+1$ zawierający węzły $x_0, \ldots, x_n$,
		\item \texttt{fx} -- wektor długości $n+1$ zawierający ilorazy różnicowe.
	\end{enumerate}
	\textbf{Oczekiwany wynik:}
	\begin{enumerate}[]
		\item \texttt{a} -- wektor długości $n+1$ zawierający obliczone współczynniki postaci naturalnej.
	\end{enumerate}
	\textbf{Opis:}\\
	Przypomnijmy, że wartości $d_0, d_1, \ldots, d_n$ są współczynnikami wielomianu interpolacyjnego w postaci Newtona. Punktem wyjściowym to wyprowadzenia algorytmu będzie uogólniony algorytm Hornera. Najpierw jednak zapiszmy wielomian interpolacyjny w postaci Newtona: 
	$$ p(x) = \underbrace{d_0 + (x-x_0)(\underbrace{d_1 + (x-x_1)
	(d_2 + \ldots + (x-x_{n-2})(\overbrace{d_{n-1}+(x-x_{n-1})\underbrace{d_n}_{W_n}}^{W_{n-1}}))}_{W_1}}_{W_0}\ldots)$$

	\noindent Idea jest bardzo podobna do wyprowadzania uogólnionego algorytmu Hornera w poprzednim zadaniu. Teraz przyjrzyjmy się zaznaczonym wielomianom $W_k$. Zauważmy też, że ich stopnie rosną (patrząc od góry do dołu).
	\begin{align*}
	W_n(x) & = d_n\nonumber \\
	W_{n-1}(x) &= d_{n-1} + (x-x_{n-1})W_n \nonumber \\
	\ldots & = \ldots \\
	W_k(x) &= d_{k} + (x - x_{k})W_{k+1} \text{ dla } 0\le k<n \nonumber \\
	\ldots & = \ldots \\
	W_0(x) &= p_0(x) \nonumber \\
	\end{align*}
	Dodatkowo mamy też, że $deg(W_k) = deg(W_{k+1}) + 1$. Obie te obserwacje są kluczowe dla wyprowadzenia algorytmu. Przypomnijmy, że wartości $d_0, d_1, \ldots, d_n$ oraz $x_0, \ldots, x_n$ są danymi, a więc są znane.
	Zastanówmy się jak obliczyć współczynniki wielomianu $W_{n-1}$ w postaci naturalnej. Mamy $deg(W_{n-1}) = deg(W_n) + 1 = 0 + 1 = 1$, a zatem $W_{n-1}$ możemy ogólnie zapisać jako $W_{n-1}(x) = a_{0}x^0 + a_{1}x^1$. Z drugiej strony patrząc na tabelę wyżej możemy go zapisać jako $$W_{n-1}(x) = d_{n-1} + (x - x_{n-1})W_n = d_{n-1} + (x - x_{n-1})d_n = d_{n-1} + xd_n - x_{n-1}d_n = 
	(d_{n-1} - d_nx_{n-1})x^0 + (d_{n})x^1 $$
	Otrzymaliśmy współczynniki naturalne wielomianu $W_{n-1}$. Konkretniej mamy, że $a_0=d_{n-1} - d_nx_{n-1}$ i $a_1=d_{n}$. Zróbmy to samo dla $W_{n-2}$. $$W_{n-2}(x)=d_{n-2}+(x-x_{n-2})W_{n-1} =d_{n-2}+(x-x_{n-2})(a_{0}x^0 + a_{1}x^1) = $$
	$$(d_{n-2} - x_{n-2}a_0)x^0 + (a_0-a_1x_{n-2})x^1 + a_1x^2$$
	Obliczyliśmy współczynniki $W_{n-2}$ w postaci naturalnej. Jeśli ten wielomian zapiszemy jako $W_{n-2} = b_0x^0+b_1x^1+b_2x^2$ to współczynniki będą następujące: $b_0 = d_{n-2} - x_{n-2}a_0$, $b_1=a_0-a_1x_{n-2}$, $b_2=a_1$.
	Widzimy zatem, że współczynniki przy najwyższej potędze wielomianów $W_{n-1}$ i $W_{n-2}$ są sobie równe. Ogólnie mamy, że współczynniki przy najwyższej potędze dla wielomianów $W_{k}$ i $W_{k-1}$ są sobie równe. Inna kluczowa obserwacja to fakt, że licząc współczynniki naturalne wielomianu $W_{n-2}$ korzystaliśmy tylko z danych i ze współczynników naturalnych wielomianu $W_{n-1}$. Podobnie obliczając współczynniki wielomianu $W_{n-1}$ korzystaliśmy tylko z danych i współczynników wielomianu $W_{n}$. Widzimy zatem, że współczynniki naturalne wielomianu $W_k$ jesteśmy w stanie obliczyć znając współczynniki wielomianu $W_{k+1}$. Oznacza to, że możemy to zrobić używając jednej tablicy. Ponadto jesteśmy w stanie zrobić to w czasie $O(n)$. W szczególności możemy obliczyć współczynniki wielomianu $W_0=p_0$ w postaci naturalnej licząc kolejno współczynniki wielomianów $W_n$, $W_{n-1}$, $\ldots$, $W_1$, $W_0$ każdy w czasie $O(n)$ co daje łączny czas $O(n^2)$. Teraz dla podsumowania pseudokod.\\
	\noindent\textbf{Pseudokod algorytmu}\\
	\begin{algorithm}[h]
		\DontPrintSemicolon
		\SetKw{KwDownTo}{downto}
		\SetKwProg{Fun}{function}{}{}
		\SetKwFunction{LEN}{length}
		\SetKwFunction{F}{naturalna}
		\Fun{\F{$x$, $fx$}} {
			$n \gets \LEN{fx}$-1\;
			$a[n] \gets fx[n]$\;
			\For{$i \gets n-1$ \KwDownTo $0$} {
				$a[i] \gets fx[i] - a[i+1] \times x[i]$\; 
				\For{$j \gets i+1$ \KwTo $n-1$} {
					$a[j] \gets a[j] - a[j+1] * x[i]$\; 	
				}	
			}
			\KwRet $a$\;
		}
		
		\caption{Obliczanie współczynników naturalnych wielomianu interpolacyjnego.}
	\end{algorithm}	
	
	
	
	
	\section*{Zadanie 4}

	Celem zadania było napisanie funkcji interpolującej zadaną funkcję $f(x)$ w przedziale $[a, b]$ za pomocą wielomianu interpolacyjnego stopnia $n$ w postaci Newtona, a także rysującej wykresy funkcji $f$ oraz otrzymanego wielomianu interpolacyjnego. W interpolacji funkcji należało użyć węzłów równoodległych.\\
	\textbf{Dane:}
	\begin{enumerate}[]
		\item \texttt{f} -- zadana funkcja,
		\item \texttt{a, b} --  przedział interpolacji,
		\item \texttt{n} --  stopień wielomianu interpolacyjnego.
	\end{enumerate}
	\textbf{Oczekiwany wynik:}
	\begin{enumerate}[]
		\item -- wykres funkcji $f$ oraz wielomianu interpolacyjnego w przedziale $[a,b]$.
	\end{enumerate}
	\textbf{Opis:}\\
	Na początku wyznaczyłem węzły interpolacyjne $x_1, \ldots, x_{n+1}$ w taki sposób aby odległość między nimi wynosiła $\frac{b-a}{n}$. Następnie obliczyłem wartości funkcji w tych punktach tj. $f(x_1), \ldots, f(x_{n+1})$.
	W celu obliczenia ilorazów różnicowych posłużyłem się funkcją z zadania pierwszego -- \texttt{ilorazyRoznicowe}. Następnie użyłem funkcji z zadania drugiego tj. \texttt{warNewton} do obliczenia wartości wielomianu interpolacyjnego w potrzebnych punktach. W celu uzyskania dokładniejszego wykresu, punkty dla których rysowałem wykres musiałem zagęścić. Zrobiłem to mnożąc dane $n$ razy obrany przeze mnie parametr gęstości równy 40. Dzięki temu uzyskałem dokładniejsze wykresy.
	
	
	\section*{Zadanie 5}
	Celem zadania było przetestowanie funkcji \texttt{rysujNnfx(f,a,b,n)} (z zadania 4) na następujących przykładach:
	\begin{enumerate}[(a)]
		\item $f(x) = e^x$, $[a, b] = [0,1]$, $n \in \{5,10,15\}$,
		\item $f(x) = x^2\sin{x}$, $[a, b] = [-1,1]$, $n \in \{5,10,15\}$.
	\end{enumerate}
	Poniżej narysowane wykresy dla obu funkcji.
	
		\begin{figure}[!htbp]
	\centering
	\subfloat[1.][$n=5$]{\includegraphics[width=0.5\textwidth]{plots/task5plotf1_5.png}} \hfill
	\subfloat[2.][$n=10$]{\includegraphics[width=0.5\textwidth]{plots/task5plotf1_10.png}} \hfill
	\subfloat[3.][$n=15$]{\includegraphics[width=0.5\textwidth]{plots/task5plotf1_15.png}} \hfill
	\caption*{Wykres funkcji $e^{x}$ i jej wielomianu interpolacyjnego dla danego stopnia $n$}
\end{figure}		

\begin{figure}[!htbp]
	\centering
	\subfloat[1.][$n=5$]{\includegraphics[width=0.5\textwidth]{plots/task5plotf2_5.png}} \hfill
	\subfloat[2.][$n=10$]{\includegraphics[width=0.5\textwidth]{plots/task5plotf2_10.png}} \hfill
	\subfloat[3.][$n=5$]{\includegraphics[width=0.5\textwidth]{plots/task5plotf2_15.png}} \hfill
	\caption*{Wykres funkcji $x^2\sin{x}$ i jej wielomianu interpolacyjnego dla danego stopnia $n$}
\end{figure}	

Na przykładach tych funkcji widać, że wybranie równoodległych węzłów dało bardzo dokładne przybliżenia funkcji. Dla żadnego z wykresów nie zaobserwowano rozbieżności. Kolejna rzecz warta zaobserwowania to fakt, że dla wszystkich wartości n funkcje były bardzo dobrze przybliżone.



	\clearpage
	
	\section*{Zadanie 6}
	
	Celem zadania było 



\end{document}