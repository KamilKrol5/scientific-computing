\documentclass[]{article}
\usepackage[utf8]{inputenc}
\usepackage{polski}
\usepackage{listings}
\usepackage[usenames,dvipsnames]{xcolor}
\usepackage{geometry}
\geometry{
	a4paper,
	left=25mm,
	top=25mm,
}
%%\hyphenchar\font=-1

\title{
	Sprawozdanie \\
	\large 
	Obliczenia naukowe - lista 1}
\author{Kamil Król}
\date{}

%% julia definition and coloring
\lstdefinelanguage{Julia}%
{morekeywords={abstract,break,case,catch,const,continue,do,else,elseif,%
		end,export,false,for,function,immutable,import,importall,if,in,%
		macro,module,otherwise,quote,return,switch,true,try,type,typealias,%
		using,while},%
	sensitive=true,%
	alsoother={$},%
	morecomment=[l]\#,%
	morecomment=[n]{\#=}{=\#},%
	morestring=[s]{"}{"},%
	morestring=[m]{'}{'},%
}[keywords,comments,strings]%

\lstset{%
	language         = Julia,
	basicstyle       = \ttfamily,
	keywordstyle     = \bfseries\color{blue},
	stringstyle      = \color{magenta},
	commentstyle     = \color{ForestGreen},
	showstringspaces = false,
}
%% end of julia definition and coloring

\begin{document}
	
	\maketitle
	
	\section*{Zadanie 1}
	
	\subsection*{MachEps}
	Epsilonem maszynowym macheps (ang. machine epsilon) nazywamy najmniejszą liczbę macheps większą od 0 taką, że fl(1.0 + macheps)\textgreater 1.0.
	
	W celu wyznaczenia metodą iteracyjną \textit{macheps} zgodnego z powyższą definicją napisałem program, a jego wyniki zamieściłem w poniższej tabeli. Zgodnie z treścią zadania program uruchomiłem dla typów Float16, Float32 oraz Float64 i porównałem z wartościami zwracanymi przez funkcję \textit{eps} dla każdego z typów.
	
	\begin{table}[h!]
		\centering
		\label{tab:table1}
		\begin{tabular}{|c|c|c|c|}
			\multicolumn{4}{c}{Iteracyjne wyznaczanie macheps}\\
			\hline
			& obliczony \textit{macheps} & eps(type) & wartość z pliku float.h \\
			\hline
			Float16 & 0.000977 & 0.000977 & xd \\
			\hline
			Float32 & 1.1920929e-7 & 1.1920929e-7 & xd \\
			\hline
			Float64 & 2.220446049250313e-16 & 2.220446049250313e-16 & xd \\
			\hline
		\end{tabular}
	\end{table}
	
	Okazało się, że wartości \textit{macheps} wyznaczone przeze mnie są równe wartościom zwracanym przez wbudowaną w język Julia funkcją \textit{eps}.
	
	W treści zadania pojawia się pytanie: jaki związek ma liczba \textit{macheps} z precyzją arytmetyki (oznaczaną na wykładzie przez $\epsilon$)? W celu odpowiedzi na to pytanie przytoczę najpierw definicję precyzji arytmetyki - $\epsilon$. Jest to największy błąd względny reprezentacji liczby jaki możemy popełnić i dla liczb reprezentowanych zgodnie ze standardem IEEE-754 wyraża się on wzorem: \(2^{-t}\). Podstawiając do wzoru dla arytmetyki Float32 mamy: \[\epsilon = 2^{-24} = 0.5 \cdot 2 ^{-23} = \frac{1}{2} \cdot \textit{macheps}\]
	Wartość \textit{macheps} dla Float32 w tabeli tj. 1.1920929e-7 jest zaokrąglona. Jej dokładna wartość wynosi: 1.1920928955078125e-7 co jest równe \(2 ^{-23}\) (stąd równość). \textit{Macheps} jest w komputerze przechowywany dokładnie.
	Wykonując to rozumowanie dla wszystkich typów widzimy zgodność i otrzymujemy zależność: \mbox{\(\textit{macheps} = 2\epsilon\)}.
	
	
	\subsection*{Eta}
	Kolejnym zadaniem jest wyznaczenie liczby \textit{eta} takiej, że \textit{eta} \textgreater\space0.0 dla wszystkich typów zmiennopozycyjnych Float16, Float32, Float64.
	Wyniki napisanego przeze mnie programu, który iteracyjnie wyznacza te liczby, umieściłem w poniższej tabeli. Ponadto wartości otrzymanych liczb \textit{eta} porównałem z wartościami zwracanymi przez funkcje: \mbox{\textit{nextfloat}(Float16(0.0))}, \mbox{\textit{nextfloat}(Float32(0.0))}, \mbox{\textit{nextfloat}(Float64(0.0)}
	
	\begin{table}[h!]
		\centering
		\label{tab:table1}
		\begin{tabular}{|c|c|c|}
			\multicolumn{3}{c}{Iteracyjne wyznaczanie eta}\\
			\hline
			& obliczona \textit{eta} & nextfloat(type(0))  \\
			\hline
			Float16 & 6.0e-8 & 6.0e-8 \\
			\hline
			Float32 & 1.0e-45 & 1.0e-45 \\
			\hline
			Float64 & 5.0e-324 & 5.0e-324 \\
			\hline
		\end{tabular}
	\end{table}

	Wartości obliczone przeze mnie okazały się takie same jak zwrócone przez funkcje wbudowane w język Julia.
	Kolejnym pytaniem jest: Jaki związek ma liczba \textit{eta} z liczbą $MIN_{sub}$?\newline
	$MIN_{sub}$ jest najmniejszą liczbą zdenormalizowaną (subnormalną), tzn. taką gdzie cecha liczby jest wypełniona zerami. Inaczej najmniejsza możliwa do przechowania w danym systemie liczba. Liczba \textit{eta} jest równa liczbie $MIN_{sub}$, są one tożsame. Są to liczby tak małe, że nie da się ich pomniejszyć manipulując cechą.
	\newline
	Innym pytaniem z treści zadania jest: co zwracają funkcje \textit{floatmin}(Float32) i \textit{floatmin}(Float64) i jaki jest związek zwracanych wartości z liczbą $MIN_{nor}$?\newline
	Funkcje te zwracają najmniejsze liczby znormalizowane dla danego typu, a jest to dokładnie $MIN_{nor}$. Liczby znromalizowane to takie gdzie w mantysie zakładamy niepisaną jedynkę, tzn. wartość mantysy '0011...' oznacza '1.0011...' (inaczej niż w subnormalnych) i cecha nie jest zerem. Inaczej są to liczby, które można pomniejszyć zmniejszając wartość cechy.
	\subsection*{Liczba MAX}
	Kolejnym zadaniem do zrobienia było wyznaczenie (iteracyjnie) liczby \textit{MAX} dla wszystkich typów zmiennopozycyjnych Float16, Float32, Float64 i porównanie wyników z wartościami zwracanymi przez funkcje:
	\mbox{\textit{floatmax}(Float16)}, \mbox{\textit{floatmax}(Float32)}, \mbox{\textit{floatmax}(Float64)} oraz z danymi zawartymi w pliku nagłówkowym float.h dowolnej instalacji języka C. Liczbę \textit{MAX} interpretuję jako największą wartość jaką można przechować w danym typie zmiennoprzecinkowym. Przy wyznaczaniu tej wartości musiałem pamiętać aby mantysa była wypełniona jedynkami. By to uzyskać postanowiłem wziąć liczbę \textit{zaraz przed} liczbą 2.0 czyli 2.0 - \textit{macheps}. Ten rezultat mogłem uzyskać też biorąc liczbę \textit{zaraz przed} 1.0, wtedy byłoby to \(1.0 - \frac{\textit{macheps}}{2}\).
	
	\begin{table}[h!]
		\centering
		\label{tab:table1}
		\begin{tabular}{|c|c|c|c|}
			\multicolumn{4}{c}{Iteracyjne wyznaczanie liczby \textit{MAX}}\\
			\hline
			& Obliczony \textit{MAX} & maxfloat(type) & wartość z pliku float.h \\
			\hline
			Float16 & 6.55e4 & 6.55e4 & xd \\
			\hline
			Float32 & 3.4028235e38 & 3.4028235e38 & xd \\
			\hline
			Float64 & 1.7976931348623157e308 & 1.7976931348623157e308 & xd \\
			\hline
		\end{tabular}
	\end{table}
	
	Ponownie wartości wyznaczone przeze mnie okazały się takie same jak te wyznaczone przez funkcje z języka Julia.
	
	\section*{Zadanie 2}
	
	Zadanie to dotyczyło sprawdzenia eksperymentalnie w języku Julia
	słuszności tezy Kahana dla wszystkich typów zmiennopozycyjnych \mbox{Float16}, \mbox{Float32}, \mbox{Float64}. Wyniki programu zamieściłem w poniższej tabeli:
	\begin{table}[h!]
		\centering
		\label{tab:table1}
		\begin{tabular}{|c|c|c|}
			\hline
			& Obliczony eps wg. wzoru Kahana & Wartość funkcji eps(type)\\
			\hline
			Float16 & -2.220446049250313e-16 & 2.220446049250313e-16 \\
			\hline
			Float32 & 1.1920929e-7 & 1.1920929e-7\\
			\hline
			Float64 &-2.220446049250313e-16 &2.220446049250313e-16\\
			\hline
		\end{tabular}
	\end{table}
	
	Ponadto mój program sprawdzał, czy wartości bezwzględne otrzymanych eps są równe i okazało się że tak jest. Wnioskiem z doświadczenia jest to, że teza Kahana jest słuszna - macheps można obliczyć stosując zaproponowany przez niego wzór. W celu otrzymania macheps należy na wynik otrzymany ze wzoru \(3(4/3-1)-1\) nałożyć wartość bezwzględną. 
	
	\clearpage
	\section*{Zadanie 3} 
	
	Celem tego zadania było eksperymentalne sprawdzenie, że w arytmetyce \mbox{Float64} liczby zmiennopozycyjne są równomiernie rozmieszczone w [1, 2] z
	krokiem \(\delta = 2^{-52}\).
	Mój program zaczynał od liczby 1.0 i z każdą iteracją dodawał do niej liczbę $\delta$. W celu zidentyfikowania czy wybrany krok jest poprawny drukowałem kolejne wartości liczb i obserwowałem jak się zmieniają. Rezultaty w tabeli poniżej. 
	
		\begin{table}[h!]
		\centering
		\label{tab:table1}
		\begin{tabular}{|c|c|}
			\hline
			& bitowa reprezentacja \\
			\hline
			$1.0 + 1\delta$ & 0|01111111111|0000000000000000000000000000000000000000000000000001 \\
			\hline                    
			$1.0 + 2\delta$ & 0|01111111111|0000000000000000000000000000000000000000000000000010 \\
			\hline                    
			$1.0 + 3\delta$ & 0|01111111111|0000000000000000000000000000000000000000000000000011 \\
			\hline             
			$1.0 + 4\delta$ & 0|01111111111|0000000000000000000000000000000000000000000000000100 \\
			\hline             
			$1.0 + 5\delta$ & 0|01111111111|0000000000000000000000000000000000000000000000000101 \\
			\hline             
			$1.0 + 6\delta$ & 0|01111111111|0000000000000000000000000000000000000000000000000110 \\
			\hline             
			$1.0 + 7\delta$ & 0|01111111111|0000000000000000000000000000000000000000000000000111 \\
			\hline             
			$1.0 + 8\delta$ & 0|01111111111|0000000000000000000000000000000000000000000000001000 \\
			\hline             
		%%	$1.0 + 9\delta$ & 0|01111111111|0000000000000000000000000000000000000000000000001001 \\
		%%	\hline
			\multicolumn{2}{c}{...} \\
		\end{tabular}
	\end{table}

	Widać w kolejnych rekordach, że wybrany krok - $\delta$ jest poprawny (dla przedziału [1,2]), ponieważ wartości zmieniają się na najmniej znaczących bitach (na końcu) w sposób umożliwiający przejście przez wszystkie możliwe wartości mantysy. \newline 
	Pojawia się pytanie: Jak rozmieszczone są liczby zmiennopozycyjne w przedziale \([\frac{1}{2},1]\), a jak w przedziale \mbox{[2, 4]} i jak
	mogą być przedstawione dla rozpatrywanego przedziału? \newline
	Liczbę $\delta$ dla przedziału \([\frac{1}{2},1]\) będziemy nazywać $\delta_1$, a dla \mbox{[2, 4]}  $\delta_4$. Liczb we wszystkich trzech wspomianych przedziałach jest tyle samo (w tej reprezentacji). Np. między 1 i 2 jest tyle samo liczb co między 2 i 4, a tych jest tyle samo co w przedziale [4,8] itd. Granicami tych przedziałów są potęgi liczby 2. Jak można z tą wiedzą wyznaczyć $\delta$ dla przedziałów  \([\frac{1}{2},1]\) \mbox{i [2, 4]?} Wiemy, że $\delta$ dla przedziału [1,2] wynosi \(\delta = 2^{-52}\). Długość przedziału \([\frac{1}{2},1]\) jest 2 razy mniejsza od długości [1,2], więc naturalnym kandydatem dla liczby $\delta_1$ (dla przedziału \([\frac{1}{2},1]\)) wydaje się liczba $\delta$ podzielona przez 2. Zatem mamy \(\delta_1 = 2^{-53}\). \newline 
	Przedział \mbox{[2, 4]} ma długość dwa razy większą niż \mbox{[1, 2]}, więc kandydatem na liczbę $\delta_4$ będzie liczba 2 razy więszka od $\delta$. W związku z tym mamy \(\delta_4 = 2^{-51}\).
	Pozostaje te wartości eksperymentalnie sprawdzić. W poniższej tabeli znajdują się wyniki programu, który sprawdza liczby $\delta_1$ i $\delta_4$ w analogiczny sposób jak przy sprawdzaniu $\delta$ dla przedziału \mbox{[1,2]}.
	
	\begin{table}[h!]
	\centering
	\label{tab:table1}
		\begin{tabular}{|c|c|}
			\multicolumn{2}{c}{\(\delta_1 = 2^{-53}\)} \\
			\hline
			& bitowa reprezentacja \\
			\hline
			$0.5 + 1\delta$ &  0|01111111110|0000000000000000000000000000000000000000000000000001 \\ \hline
			$0.5 + 2\delta$ &  0|01111111110|0000000000000000000000000000000000000000000000000010 \\ \hline
			$0.5 + 3\delta$ &  0|01111111110|0000000000000000000000000000000000000000000000000011 \\ \hline
			$0.5 + 4\delta$ &  0|01111111110|0000000000000000000000000000000000000000000000000100 \\ \hline
			$0.5 + 5\delta$ &  0|01111111110|0000000000000000000000000000000000000000000000000101 \\ \hline
			$0.5 + 6\delta$ &  0|01111111110|0000000000000000000000000000000000000000000000000110 \\ \hline
			$0.5 + 7\delta$ &  0|01111111110|0000000000000000000000000000000000000000000000000111 \\ \hline
			$0.5 + 8\delta$ &  0|01111111110|0000000000000000000000000000000000000000000000001000 \\ \hline
			$0.5 + 9\delta$ &  0|01111111110|0000000000000000000000000000000000000000000000001001 \\ \hline
		\end{tabular}
	\end{table}
	\begin{table}[h!]
		\centering
		\label{tab:table1}
			\begin{tabular}{|c|c|}
			\multicolumn{2}{c}{\(\delta_4 = 2^{-51}\)} \\
			\hline
			& bitowa reprezentacja \\
			\hline
			$2.0 + 1\delta$ &  0|10000000000|0000000000000000000000000000000000000000000000000001 \\ \hline
			$2.0 + 2\delta$ &  0|10000000000|0000000000000000000000000000000000000000000000000010 \\ \hline
			$2.0 + 3\delta$ &  0|10000000000|0000000000000000000000000000000000000000000000000011 \\ \hline
			$2.0 + 4\delta$ &  0|10000000000|0000000000000000000000000000000000000000000000000100 \\ \hline
			$2.0 + 5\delta$ &  0|10000000000|0000000000000000000000000000000000000000000000000101 \\ \hline
			$2.0 + 6\delta$ &  0|10000000000|0000000000000000000000000000000000000000000000000110 \\ \hline
			$2.0 + 7\delta$ &  0|10000000000|0000000000000000000000000000000000000000000000000111 \\ \hline
			$2.0 + 8\delta$ &  0|10000000000|0000000000000000000000000000000000000000000000001000 \\ \hline
			$2.0 + 9\delta$ &  0|10000000000|0000000000000000000000000000000000000000000000001001 \\ \hline
		\end{tabular}
	\end{table}

	Widać, że wartości $\delta_1$ i $\delta_4$ są poprawne z tego samego powodu co poprzednio - wartości zmieniają się na najmniej znaczących bitach (na końcu) w sposób umożliwiający przejście przez wszystkie możliwe wartości mantysy.
	
	\section*{Zadanie 4}

	Celem tego zadania było eksperymentalne znalezienie w arytmetyce Float64 (double) najmniejszej liczby zmiennopozycyjnej x w przedziale (1,2) takiej, że \(x \cdot \frac{1}{x} \neq 1\). 
	Wynik mojego programu to: 1.000000057228997. Reprezentacja tej liczby w arytmetyce Float64 to: \newline 
	\mbox{0 01111111111 0000000000000000000000001111010111001011111100101010}.\newline
	Mantysa zinterpretowana jako samodzielna liczba to 257736490. Oznacza to, że program musiał wykonać aż 257736490 - 1 = 257736480 iteracji.
	\colorbox{BurntOrange}{TO DO} Wnioski?\newline
	
	\section*{Zadanie 5}
	Zadanie to polegało na eksperymentalnym obliczeniu iloczynu skalarnego dwóch wektorów x i y. \newline
	\mbox{x = [2.718281828, -3.141592654, 1.414213562, 0.5772156649, 0.3010299957]}\newline
	\mbox{y = [1486.2497, 878366.9879, -22.37492, 4773714.647, 0.000185049].}\newline
	W treści zadania są podane 4 algorytmy: \newline a (forward),b (backward),c (descending),d (ascending), które zaimplementowałem. Algorytmy są opisane w treści zadania, więc nie będę ich tu opisywać. Kolejnym krokiem w zadaniu było porównanie otrzymanych wartości z wartościa prawdziwą, zrealizowałem to licząc błąd bezwzględny i względny. Przypomnijmy - wartość dokładna (z dokładnością do 15 cyfr, według treści zadania) wynosi \(-1.00657107\cdot10^{-11}\). Wyniki mojego programu znajdują się w tabeli poniżej:
	
	\begin{table}[h!]
	\centering
	\label{tab:table1}
		\begin{tabular}{|c|c|c|c|}
			\multicolumn{4}{c}{Dla Float64}\\
			\hline
			alogrytm & obliczona wartość & błąd bezwzględny & błąd względny\\
			\hline
			Forward & 1.0251881368296672e-10 & 1.1258452438296672e-10 & 11.184955313981627 \\ \hline
			Backward & -1.5643308870494366e-10 & 1.4636737800494365e-10 & 14.541186645165915 \\ \hline
			Descending & 0.0 & 1.00657107e-11 & 1.0 \\ \hline
			Ascending & 0.0 & 1.00657107e-11 & 1.0 \\ \hline
		\end{tabular}
	\end{table}

	\begin{table}[h!]
	\centering
	\label{tab:table1}
		\begin{tabular}{|c|c|c|c|}
			\multicolumn{4}{c}{Dla Float32}\\
			\hline
			alogrytm & obliczona wartość & błąd bezwzględny & błąd względny\\
			\hline
			Forward & -0.4999443 & 0.49994429944939167 & 4.9668057661282845e10 \\ \hline
			Backward & -0.4543457 & 0.4543457031149343 & 4.51379655800096e10 \\ \hline
			Descending & -0.5 & 0.4999999999899343 & 4.967359135306107e10 \\ \hline
			Ascending & -0.5 & 0.4999999999899343 & 4.967359135306107e10 \\ \hline
		\end{tabular}
	\end{table}
	
	\clearpage
	Patrząc na wyniki nasuwa się kilka ważnych wniosków. Wartości obliczone przez program nie są równe wartości dokładnej bez względu na użyty algorytm. Ponadto różnice w wartościach obliczonych przez poszczególne algorytmy okazały się kompletnie nieintuicyjne. Algorytm c (descending), który wydaje się \textit{najgorszy} okazał się widocznie lepszy od algorytmu a i b (dla Float64) i na dodatek dał taki sam wynik jak algorytm d (ascending) (dla Float64 i Float32), który wydaje się \textit{najlepszy}. Przy czym przez \textit{najlepszy/najgorszy} rozumiem taki który oblicza wartość odpowiednio najdokładniej/najmniej dokładnie. \newline
	Inną rzeczą wartą zauważenia jest też fakt, że różnice w błędach dla dwóch testowanych typów okazały się bardzo duże - błędy dla Float64 były znacznie mniejsze.
	Kolejnym wnioskiem jest to, że licząc iloczyn skalarny dla Float64 otrzymaliśmy w 2 przypadkach wartość 0.0, co nieuważnemu użytkownikowi dałoby podstawy żeby twierdzić, że wekotry x i y są prostopadłe, kiedy w rzeczywistości tak nie jest. 

	\section*{Zadanie 6}
	
	Celem zadania było policzenie i porównanie wartości dwóch matematycznie równych funkcji (\textit{f} i \textit{g}) dla argumentów: \(8^{-1}, 8^{-2}, 8^{-3}, ...\) i określenie które z otrzymanych wyników są wiarygodne.
	\[f(x) = \sqrt{x^2 + 1} - 1\]
	\[g(x) = \frac{x^2}{\sqrt{x^2 + 1} + 1}\]
	W tabeli poniżej znajdują się wyniki mojego programu.
	\begin{table}[!h]
		\centering
		\label{tab:table1}
		\begin{tabular}{|c|c|c|c|}
			\hline
			x & f(x) & g(x) & f = g \\ \hline
			$8^{-1}$ & 0.0077822185373186414 & 0.0077822185373187065 & false \\ \hline
			$8^{-2}$ & 0.00012206286282867573 & 0.00012206286282875901 & false \\ \hline
			$8^{-3}$ & 1.9073468138230965e-6 & 1.907346813826566e-6 & false \\ \hline
			$8^{-4}$ & 2.9802321943606103e-8 & 2.9802321943606116e-8 & false \\ \hline
			$8^{-5}$ & 4.656612873077393e-10 & 4.6566128719931904e-10 & false \\ \hline
			$8^{-6}$ & 7.275957614183426e-12 & 7.275957614156956e-12 & false \\ \hline
			$8^{-7}$ & 1.1368683772161603e-13 & 1.1368683772160957e-13 & false \\ \hline
			$8^{-8}$ & 1.7763568394002505e-15 & 1.7763568394002489e-15 & false \\ \hline
			$8^{-9}$ & 0.0 & 2.7755575615628914e-17 & false \\ \hline
			$8^{-10}$ & 0.0 & 4.336808689942018e-19 & false \\ \hline
			$8^{-11}$ & 0.0 & 6.776263578034403e-21 & false \\ \hline
			$8^{-12}$ & 0.0 & 1.0587911840678754e-22 & false \\ \hline
			$8^{-13}$ & 0.0 & 1.6543612251060553e-24 & false \\ \hline
			$8^{-14}$ & 0.0 & 2.5849394142282115e-26 & false \\ \hline
			$8^{-15}$ & 0.0 & 4.0389678347315804e-28 & false \\ \hline
			$8^{-16}$ & 0.0 & 6.310887241768095e-30 & false \\ \hline
			$8^{-17}$ & 0.0 & 9.860761315262648e-32 & false \\ \hline
			$8^{-18}$ & 0.0 & 1.5407439555097887e-33 & false \\ \hline
			$8^{-19}$ & 0.0 & 2.407412430484045e-35 & false \\ \hline
			\multicolumn{4}{c}{$\vdots$} \\ \hline
			$8^{-23}$ & 0.0 & 1.4349296274686127e-42 & false \\ \hline
			$8^{-24}$ & 0.0 & 2.2420775429197073e-44 & false \\ \hline
			$8^{-25}$ & 0.0 & 3.503246160812043e-46 & false \\ \hline
			$8^{-26}$ & 0.0 & 5.473822126268817e-48 & false \\ \hline
			$8^{-27}$ & 0.0 & 8.552847072295026e-50 & false \\ \hline
			$8^{-28}$ & 0.0 & 1.3363823550460978e-51 & false \\ \hline
			$8^{-29}$ & 0.0 & 2.088097429759528e-53 & false \\ \hline
			$8^{-30}$ & 0.0 & 3.2626522339992623e-55 & false \\ \hline
		\end{tabular}
	\end{table}

	\clearpage
	Widać, że dla żadnego z badanych argumentów funkcje nie są równe. Widać również, że począwszy od \(8^{-9}\) funkcja \textit{f} daje wartość równą 0.0 podczas kiedy funkcja \textit{g} nie. Zastanówmy się skąd mogła się wziąć ta wartość 0.0. Rozważmy sytuację kiedy liczba $x \ll 1.0$. W takim przypadku może wystąpić zjawisko tzw. pochłonięcia. Przy wykonywaniu dodawania cechy liczb są wyrównywane i w przypadku znacznej różnicy w wartościach liczb, cyfry liczby $x^2$ mogą zostać całkiem pochłonięte gdyż przy takiej samej cesze mantysy nie są w stanie uwzględnić cyfr znaczących dla obu liczb. W rezultacie pod pierwiastkiem pojawia się wartość 1.0. Stąd po wykonaniu dodawania otrzymujemy \(\sqrt{1} - 1 = 0\).\newline
	Zobaczmy jak to wygląda dla x = \(8^{-9}\). Pierwsze co robimy to podnosimy x do kwadratu. Mamy:
	 \(({8^{-9}})^2 = 8^{-18} = ({2^3})^{-18} = 2^{3 \cdot (-18)} = 2^{-54} = x_0\). Nasza mantysa może przechować maksymalnie 52 cyfry. Reprezentacje liczb, które będziemy do siebie dodawać są dokładne. (Przez zapis 000...0 rozumiemy 52 zera)
	 \(1.0 = 2^0 \cdot 1.000...0;\space\space2^{-54} = 2^{-54} \cdot 1.000...0\). (W obu przypadkach mantysy wypełnione zerami). Teraz chcemy wyrównać cechy do większej z nich, zatem mantysę (wraz z niepisaną jedynką z przodu - 1.mantysa) liczby $2^{-54}$ przesuwamy o 54 miejsca. Teraz przystępujemy do dodawania (dodajemy w dwa razy większej precyzji) $x^2 + 1.0 = 1.\underbrace{000...001}_{54 cyfry}$.
	 W rezultacie ostatnia jedynka uciekła poza 52. pozycję i zostanie ucięta - otrzymujemy 1.000...0 - same zera po przecinku. Następnie liczmy z tego pierwiastek i odejmujemy od tego 1.0: \(\sqrt{1.0} - 1.0 = 1.0 - 1.0 = 0.0\). Stąd właśnie wzięło się nasze zero w wyniku.\newline
	 Rodzi się pytanie, dlaczego funkcja \textit{g} nie zachowała się tak jak \textit{f}. Widać, że w mianowniku wystąpi podobny problem - cały pierwiastek w pewnym momencie będzie równy 1.0, a co za tym idzie wartość całego mianownika przyjmie wartość 2.0. (Warto nadmienić, że dzielenie przez 2 jest wykonywane dokładnie). Jednak pozostanie tu licznik, który nadal będzie wpływał na wynik funkcji. Zwróćmy uwagę na to, że w momencie w którym pochłonęliśmy x obliczając wartość \textit{f(x)} straciliśmy całkiem informację o argumencie - jego wartość straciła znaczenie i nie mogła już wpłynąć na wartość funkcji. Inaczej jest z fukcją \textit{g}, w momencie pochłonięcia x-a w mianowniku, x nie stracił możliwości wpływania na wartość funkcji. Podsumowując, funkcja \textit{g} jest odporniejsza na zjawisko pochłaniania niż funkcja \textit{f}. Stąd wartości funkcji \textit{g} są bardziej wiarygodne.
	\clearpage
	\section*{Zadanie 7}
	
	Ostatnie zadanie polegało na obliczeniu przybliżonej wartości funkcji \(f(x) = sin(x) + cos(3x)\) w punkcie $x_0 = 1$ korzystając ze wzoru:
	\[f'(x) \approx \widetilde{f'}(x) = \frac{f(x_0 + h) - f(x_0)}{h}\]
	Ponadto dla \(h = 2^{-n} (n = 0,1,2,3,4,...,54)\) trzeba było obliczyć błędy  \(|f'(x) - \widetilde{f'}(x)|\). Dokładną wartość pochodnej funkcji możemy obliczyć z matematycznego wzoru: \(f'(x) = cos(x) - 3sin(3x)\). Wyniki działania mojego programu znajdują się w tabeli poniżej.
	
	
	\begin{table}[!h]
		\centering
		\label{tab:table1}
		\begin{tabular}{|c|c|c|c|}
			\hline
			n & $\widetilde{f'}(1)$ & \(|f'(x) - \widetilde{f'}(x)|\) & h+1\\ \hline
			0 & 2.0179892252685967 & 1.9010469435800585 & 2.0\\ \hline
			1 & 1.8704413979316472 & 1.753499116243109 & 1.5\\ \hline
			2 & 1.1077870952342974 & 0.9908448135457593 & 1.25\\ \hline
			3 & 0.6232412792975817 & 0.5062989976090435 & 1.125\\ \hline
			4 & 0.3704000662035192 & 0.253457784514981 & 1.0625\\ \hline
			5 & 0.24344307439754687 & 0.1265007927090087 & 1.03125\\ \hline
			6 & 0.18009756330732785 & 0.0631552816187897 & 1.015625\\ \hline
			\multicolumn{4}{c}{$\cdots$} \\ \hline
			26 & 0.11694233864545822 & 5.6956920069239914e-8 & 1.0000000149011612\\ \hline
			27 & 0.11694231629371643 & 3.460517827846843e-8 & 1.0000000074505806\\ \hline
			28 & 0.11694228649139404 & 4.802855890773117e-9 & 1.0000000037252903\\ \hline
			29 & 0.11694222688674927 & 5.480178888461751e-8 & 1.0000000018626451\\ \hline
			30 & 0.11694216728210449 & 1.1440643366000813e-7 & 1.0000000009313226\\ \hline
			31 & 0.11694216728210449 & 1.1440643366000813e-7 & 1.0000000004656613\\ \hline
			32 & 0.11694192886352539 & 3.5282501276157063e-7 & 1.0000000002328306\\ \hline
			33 & 0.11694145202636719 & 8.296621709646956e-7 & 1.0000000001164153\\ \hline
			34 & 0.11694145202636719 & 8.296621709646956e-7 & 1.0000000000582077\\ \hline
			35 & 0.11693954467773438 & 2.7370108037771956e-6 & 1.0000000000291038\\ \hline
			36 & 0.116943359375 & 1.0776864618478044e-6 & 1.000000000014552\\ \hline
			\multicolumn{4}{c}{$\cdots$} \\ \hline
			44 & 0.1171875 & 0.0002452183114618478 & 1.0000000000000568\\ \hline
			45 & 0.11328125 & 0.003661031688538152 & 1.0000000000000284\\ \hline
			46 & 0.109375 & 0.007567281688538152 & 1.0000000000000142\\ \hline
			47 & 0.109375 & 0.007567281688538152 & 1.000000000000007\\ \hline
			48 & 0.09375 & 0.023192281688538152 & 1.0000000000000036\\ \hline
			49 & 0.125 & 0.008057718311461848 & 1.0000000000000018\\ \hline
			50 & 0.0 & 0.11694228168853815 & 1.0000000000000009\\ \hline
			51 & 0.0 & 0.11694228168853815 & 1.0000000000000004\\ \hline
			52 & -0.5 & 0.6169422816885382 & 1.0000000000000002\\ \hline
			53 & 0.0 & 0.11694228168853815 & 1.0\\ \hline
			54 & 0.0 & 0.11694228168853815 & 1.0\\ \hline
		\end{tabular}
	\end{table}
	
	Widać, że dla n = 53 i 54 wartość $\widetilde{f'}(1)$ wyniosła 0 (dla dalszych wartości też tak jest), a wartość $h + 1 = 1$. Patrząc na definicję \textit{macheps} z zadania 1. możemy stwierdzić, że liczba h wraz ze wzrostem liczby n zbliżała się do \textit{macheps}. (Zwróćmy uwagę, że dla $n = 52$, h wyniosło $2^{-52} = \textit{macheps}$ i na to, że wartość $x_0 = 1$ jest tu kluczowa). Skutkowało to, że od wartości $n = 53$ wyrażenie $f(1 + h) - f(1)$ przybierało postać $f(1) - f(1) = 0$. Dla $n\space \textgreater\space 2^{-52}$, h zostaje pochłonięte podczas dodawania w liczniku i przez to później licznik przyjmuje wartość 0. Wyjaśnia to dlaczego wartości h+1 od pewnego momentu są równe dokładnie 1.\newline
	Kiedy przyjrzymy się dokładniej wartościom błędów bezwzględnych widzimy, że dokładność przybliżenia zaczeła się pogarszać dużo wcześniej niż przy \(2^{-53}\). Wartość dla funkcji \(2^{-28}\) wydaje się najdokładniejsza, gdyż błąd bezwzględny jest wtedy najmniejszy. Dalsze wartości mają już większy błąd.
	Uzasadnieniem dlaczego się tak dzieje jest zjawisko utraty cyfr znaczących. Zwróćmy uwagę, że kiedy komputer oblicza przybliżoną wartość pochodnej w punkcie $x_0 = 1$ według wzoru podanego powyżej, liczby w liczniku tj. $f(x_0 + h)$ i $f(x_0)$ mają wartości bardzo bliskie sobie. Kiedy następuje ich odejmowanie następuje zjawisko utraty cyfr znaczących. 
	Wyjaśnia to dlaczego malejące h przestaje od pewnego momentu zwiększać dokładność przybliżenia wartości pochodnej. Wynika z tego, że należy unikać odejmowania liczb bliskich sobie, ponieważ narażamy się na wyżej opisane zjawisko.
	
	
	
	
\end{document}